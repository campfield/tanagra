%
%
% VARIABLES AND SETTINGS
%
%
%
%  The \newcommand{\nameofcommand} is a LaTeX workhorse with many features/uses.
%    For a particular command you can only call \newcommand once, if you attempt to
%    define the same command with \newcommand anywhere in your document's components, LaTeX will error out.
%    The proper way to redifine commands is to use \renewcommand{\nameofcommand}.
%    However, to use \renewcommand you have to create the original command with \newcommand.
%
%  In this section, I'm using \newcommand to initialize a lot of variables, so to set my email for a university project
%    I could use \newcommand{\authoremail}{campfield@example.edu}.  However, if I'd want to set
%    it to a personal .com email I'd use \newcommand{\authoremail}{campfield@example.com}.
%    This gets annoying with having to comment in and out different entries of \newcommand.  So, I define
%    all of the options used in this specific file (and some used in other files but common to the APA7 doc requirements)
%    by setting \newcommands with empty values.
%    That way, any time below this initial variable creation I can simply use \renewcommand and thus when I want
%    to quickly swap values (such as what class I'm writing a paper for) I can just switch the line numbers
%    for \renewcommand{\subjectid}{LIS-S301} and \newcommand{\subjectid}{LIS-S550}.
%
%  Now, do not attempt to set all initial \newcommand entries for all of your variables that you will ever use in your
%    documents here.  This will cause this part of the document to be filled with junk variables that will
%    never be used across documents and can never be deleted because you will go "I can't delete this variable as
%    it may be used in some document I've not touched recently."  Put your variables in the options.tex files at the
%    correct levels (global (this document), source, subject, and project.  Now, you see any variables set here that don't
%    seem like they belong at the global level, I'm writing the project so I don't have to follow my own rules.
%
%
\newcommand{\authorname}{}
\newcommand{\authoremail}{}
\newcommand{\babelargs}{}
\newcommand{\biblatexoptions}{}
\newcommand{\bibliogen}{}
\newcommand{\subjectid}{}
\newcommand{\datetimeargs}{}
\newcommand{\dockeywords}{}
\newcommand{\documentclassoptions}{}
\newcommand{\documentmode}{}
\newcommand{\email}{}
\newcommand{\fillinline}{}
\newcommand{\fulltitle}{}
\newcommand{\indexgen}{}
\newcommand{\libshort}{}
\newcommand{\listfiggen}{}
\newcommand{\projectid}{}
\newcommand{\sectioncurrent}{}
\newcommand{\sectionheader}{}
\newcommand{\subsectionheader}{}
\newcommand{\sourceid}{}
\newcommand{\titlegen}{}


% For documents involving a single author (such as most student works), lets set a
%   that author's name at the top so we can use it everywhere and only override it
%   as necessary.  For this example, lets pick someone notable in the community.
\renewcommand{\authorname}{Michael P. Campfield}

%
% Show us first defining documents under our project sources.
%

% Swap out our previous sourcid++ to build this project's
%   documents
\renewcommand{\sourceid}{sources/tanagra}
\renewcommand{\subjectid}{open-source}
\renewcommand{\projectid}{latex-reference}

\renewcommand{\sourceid}{sources/lis}
\renewcommand{\subjectid}{s506-research}
\renewcommand{\projectid}{00-undef}

\renewcommand{\sourceid}{sources/lis}
\renewcommand{\subjectid}{s552-academic-lib-mgmt}
\renewcommand{\projectid}{00-undef}

\renewcommand{\sourceid}{sources/tanagra}
\renewcommand{\subjectid}{open-source}
\renewcommand{\projectid}{latex-reference}


% the following variable sets the document's overall style.  The general format and features
%   the document possesses (dual column, abstract pages, double vs. single spacing, etc)
% jou - APA journal article format
% man - formatted for submission to a journal
% stu - student document format
% doc - standard document
\renewcommand{\documentmode}{jou}
\renewcommand{\documentmode}{man}
\renewcommand{\documentmode}{doc}
\renewcommand{\documentmode}{stu}

%
% If indexgen is 'true' (no quotes) then call \printindex
%  will be called to create an index at the end of the document.
%
\renewcommand{\indexgen}{false}

%
% If bibliogen is 'true' (no quotes) then call \printbibliography
%  will be called to create an index at the end of the document.
%
\renewcommand{\bibliogen}{true}

%
% If listocoffiguresgen is set (default is in options.tex) then add the list of figures option
%
\renewcommand{\listfiggen}{false}
\renewcommand{\babelargs}{american}
\renewcommand{\datetimeargs}{datesep=/,style=iso}

%
% true/false - should we generate the title material (front-end matter)
%
\renewcommand{\titlegen}{true}

%
% Set a command that sets up a section header (not just \section) but includes the number of the section and increment
%   the value of the counter
%
\renewcommand{\sectionheader}[1]{\stepcounter{countersection}\section{Section \arabic{countersection}: #1}}
\renewcommand{\subsectionheader}[1]{\stepcounter{countersubsection}\subsection{Subsection\arabic{countersubsection}: #1}}

%
% Here we set several of the document style options such as font size.
%  the full list of these options can be found in the
%  package document PDF here: https://ctan.org/pkg/apa7
%

%
% It is important to note that the 'draft' option disables in-document
%  hyperlinking for PDF-style files but ALSO disable images from being packaged
%  into the final document.
% floatsintext instructs LaTeX to render tables and graphics within the
%  text instead at the end sections of the document.
%
\renewcommand{\documentclassoptions}{\documentmode,10pt,floatsintext,makeidx}

%
% BibLaTeX is a formatting tool for automagically processing
%  various forms of citations in a clean and portable manner.
%  We set the this as a variable for later use as a text
%  string in our document and when calling
%  \usepackage[\biblatexoptions]{biblatex} in the packages.tex
%  document.
%
\renewcommand{\biblatexoptions}{style=apa,backend=biber}

%
% \fillinline is a command for creating underlined text fields
%   where the text content is easily switched from visible to hidden (defaults to hidden)
%   but for the length of the text passed, hidden text or not, an underline section
%   will be present.
% If any value that is not 'hidden' is passed to the command, the text as argument #2 will
%   be visible, else not.
%   So:
%     \fillinline{springboard} will produce a string of 11 of underscores
%     \fillinline[]{springboard} will produce the underlined string 'springboard'
%
\renewcommand{\fillinline}[2][hidden]{\ifthenelse{\equal{#1}{hidden}}{\underline{\phantom{#2}}}{\underline{#2}}}


% Little command for quickly inserting a small block of libsum into the document that rolls forward
\newcounter{counterlibsum}
\setcounter{counterlibsum}{1}
\renewcommand{\libshort}{\lipsum[\arabic{counterlibsum}][1-10]\stepcounter{counterlibsum}}

